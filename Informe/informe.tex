\documentclass[11pt, fleqn]{article}
\usepackage[a4paper, margin=2.54cm]{geometry}
\usepackage[utf8]{inputenc}
\usepackage[spanish, mexico]{babel}
\usepackage[spanish]{layout}
\usepackage[article]{ragged2e}
\usepackage{textcomp}
\usepackage{amsmath}
\usepackage{amssymb}
\usepackage{amsfonts}
\usepackage{proof}

% Nombre de reglas

\newcommand{\rTVar}        {\rn{T-Var}}
\newcommand{\rTApp}        {\rn{T-TApp}}
\newcommand{\rTAbs}        {\rn{T-Abs}}
\usepackage{bussproofs}
\usepackage{graphicx}

\setlength{\parindent}{5pt}

\title{
    Trabajo Práctico N° 3 \\
    \large Análisis de Lenguajes de Programación}
\author{Mellino, Natalia \and Farizano, Juan Ignacio}
\date{}

\begin{document}
\maketitle
\noindent\rule{\textwidth}{1pt}

\section*{Ejercicio 1}

\begin{prooftree}

    \AxiomC{}
    \RightLabel{xd}
    \UnaryInfC{$
       x : E \rightarrow E \rightarrow E \in  \Gamma'  
    $}     

    \RightLabel{T-Var}
    \UnaryInfC{$
    \Gamma' \vdash x : E \rightarrow E \rightarrow E
    $}

    %==========================================

    \AxiomC{}
    \RightLabel{xd}
    \UnaryInfC{$
       z : E \in  \Gamma'  
    $}  

    \RightLabel{T-Var}
    \UnaryInfC{$
       \Gamma' \vdash z : E 
    $}

    \RightLabel{T-App}
    \BinaryInfC{$
        \Gamma' \vdash (x \; z) : E \rightarrow E
    $}
% ============================================================
    
    \AxiomC{}
    \RightLabel{xd}
    \UnaryInfC{$
         y : E \rightarrow E \in \Gamma'
    $}

    \RightLabel{T-Var}
    \UnaryInfC{$
        \Gamma' \vdash y : E \rightarrow E
    $}
%==========================================
    \AxiomC{}
    \RightLabel{xd}
    \UnaryInfC{$
         z : E \in \Gamma'
    $}

    \RightLabel{T-Var}
    \UnaryInfC{$
        \Gamma' \vdash z : E 
    $}

    \RightLabel{T-App}
    \BinaryInfC{$
        \Gamma' \vdash (y \; z) : E
    $}
% ==============================================================
    \RightLabel{T-App}
    \BinaryInfC{$        
        \underbrace{\Gamma, \; z : E}_{\Gamma'} \vdash
        (x \; z) \; (y \; z) : E
    $}

    \RightLabel{T-Abs}
    \UnaryInfC{$
        \underbrace{x : E \rightarrow E \rightarrow E, \; y: E \rightarrow E}_{\Gamma} \vdash 
        \lambda z : E. \; (x \; z) \; (y \; z) : E \rightarrow E
    $}

    \RightLabel{T-Abs}
    \UnaryInfC{$
        x : E \rightarrow E \rightarrow E \vdash \lambda y : E \rightarrow E. \; 
        \lambda z : E. \; (x \; z) \; (y \; z) : (E \rightarrow E) \rightarrow E \rightarrow E
    $}

    \RightLabel{T-Abs}
    \UnaryInfC{$
        \vdash \lambda x : E \rightarrow E \rightarrow E. \;  \lambda y : E \rightarrow E  \; 
        \lambda z : E. \; (x \; z) \; (y \; z) : (E \rightarrow E \rightarrow E) \rightarrow 
        (E \rightarrow E) \rightarrow E \rightarrow E 
    $} 
\end{prooftree}


%==============================================================================
%==============================================================================
%==============================================================================

\section*{Ejercicio 2}

La función \texttt{infer} retorna un valor del tipo \texttt{Either String Type}
porque en caso de que haya un error de tipo, se devuelve un String indicando
cuál fue el error. \\

\textbf{Funcionamiento de \texttt{>>=}:} se utiliza para aplicar una función
a un tipo, pero en el caso de que el argumento sea un error, éste se sigue pasando
y su es un tipo se aplica la función. \textbf{mmm esplicar megor che}

%==============================================================================
%==============================================================================
%==============================================================================

\section*{Ejercicio 5}


\begin{prooftree}

%======================================================================    
    \AxiomC{}

    \RightLabel{xd}
    \UnaryInfC{$
         x : E \in x : E 
    $}

    \RightLabel{T-Var}
    \UnaryInfC{$
         x : E \vdash x : E 
    $}

    \RightLabel{T-Abs}
    \UnaryInfC{$
        \vdash \lambda x : E. \; x : E \rightarrow E
    $}

    \RightLabel{T-Ascribe}
    \UnaryInfC{$
        (\lambda x : E. \; x) \; as \; E \rightarrow E : E \rightarrow E
    $}
%======================================================================

\AxiomC{}
    \RightLabel{xd}
    \UnaryInfC{$
        z : E \rightarrow E \vdash z : E \rightarrow E 
    $}

    \RightLabel{T-Var}
    \UnaryInfC{$
        z : E \rightarrow E \vdash z : E \rightarrow E 
    $}
%=====================================================================
    \RightLabel{T-Let}
    \BinaryInfC{$
        (let \; z = ((\lambda x : E. \; x) \; as \; E \rightarrow E) \; in \; z) : E \rightarrow E
    $}
    \RightLabel{T-Ascribe}
    \UnaryInfC{$
        \vdash (let \; z = ((\lambda x : E. \; x) \; as \; E \rightarrow E) \; in \; z) 
        \; as \; E \rightarrow E : E \rightarrow E 
    $}
\end{prooftree}


%==============================================================================
%==============================================================================
%==============================================================================

\section*{Ejercicio 7}


\begin{prooftree}

    \AxiomC{$
         t_1 \rightarrow t_1'
    $}

    \RightLabel{E-Pair$_1$}
    \UnaryInfC{$
        (t_1, \; t_2) \rightarrow (t_1', \; t_2)
    $} 
\end{prooftree}

\begin{prooftree}

    \AxiomC{$
         t_2 \rightarrow t_2'
    $}

    \RightLabel{E-Pair$_2$}
    \UnaryInfC{$
        (v, \; t_2) \rightarrow (v, \; t_2')
    $} 
\end{prooftree}

\begin{prooftree}

    \AxiomC{$
         t \rightarrow t'
    $}

    \RightLabel{E-Fst}
    \UnaryInfC{$
        fst \; t \rightarrow fst \; t'
    $} 
\end{prooftree}

\begin{prooftree}

    \AxiomC{$
         t \rightarrow t'
    $}

    \RightLabel{E-Snd}
    \UnaryInfC{$
        snd \; t \rightarrow snd \; t'
    $} 
\end{prooftree}

\begin{equation*}
    fst \; (v_1, \; v_2) \rightarrow v_1 \; \; \; \; E-Fst'
\end{equation*}

\begin{equation*}   
    snd \; (v_1, \; v_2) \rightarrow v_2 \; \; \; \; E-Snd'
\end{equation*}


%==============================================================================
%==============================================================================
%==============================================================================

\section*{Ejercicio 9}



\begin{prooftree}
    \AxiomC{}
    \RightLabel{T-Unit}
    \UnaryInfC{$
        \vdash unit \; : \; Unit
    $}

    \RightLabel{T-Ascribe}
    \UnaryInfC{$
        \vdash unit \; as \; Unit : Unit
    $}

%=======================================================
    \AxiomC{$
        x : (E, \; E) \in x : (E, \; E)
    $}

    \RightLabel{T-Var}
    \UnaryInfC{$
        x : (E, \; E) \vdash x : (E, \; E)
    $}

    \RightLabel{T-Snd}
    \UnaryInfC{$
        \vdash x : (E, \; E). \; snd \; x : E
    $}

    \RightLabel{T-Abs}
    \UnaryInfC{$
        \vdash \lambda x : (E, \; E). \; snd \; x : (E, \; E) \rightarrow E
    $}

    \RightLabel{T-Pair}
    \BinaryInfC{$
        \vdash (unit \; as \; Unit, \lambda x : (E, \; E). snd \; x) : (Unit, \; (E, \; E) \rightarrow E)
    $}

    \RightLabel{T-Fst}
    \UnaryInfC{$
        \vdash fst \; (unit \; as \; Unit, \lambda x : (E, \; E). snd \; x) : Unit
    $}
\end{prooftree}


\end{document}